\documentclass[12pt, a4paper]{article}

\usepackage[utf8]{inputenc}
\usepackage[T1]{fontenc}
\usepackage[brazil]{babel}
\usepackage{geometry}
\usepackage{enumitem}
\usepackage{hyperref}
\usepackage{fancyhdr}
\usepackage{bookmark}

\geometry{a4paper, top=3cm, bottom=3cm, left=3cm, right=2cm}
\hypersetup{
    colorlinks=true,
    linkcolor=black,
    filecolor=magenta,
    urlcolor=blue,
}

\setlength{\headheight}{16pt}

\pagestyle{fancy}
\fancyhf{}
\fancyhead[C]{Estatuto da Liga de Cibersegurança da UNICAMP}
\fancyfoot[C]{\thepage}
\renewcommand{\headrulewidth}{0.4pt}
\renewcommand{\footrulewidth}{0.4pt}

\usepackage[scaled]{helvet}
\renewcommand\familydefault{\sfdefault}
\usepackage[T1]{fontenc}

\title{Estatuto da Liga de Cibersegurança da UNICAMP}
\author{Liga de Cibersegurança da UNICAMP}
\date{\today}

\begin{document}

\maketitle
\thispagestyle{empty}
\newpage

\tableofcontents
\newpage

\section{Capítulo I – Das Disposições Preliminares}


\subsection{Art. 1º Denominação, Sede e Fins}
\begin{enumerate}[label=\S \arabic*.]
    \item A Liga de Cibersegurança da UNICAMP, doravante denominada Liga, fundada em \today, é uma sociedade civil, sem fins lucrativos, de duração ilimitada, apartidária, não religiosa e academicamente vinculada à Faculdade de Tecnologia (FT) da Universidade Estadual de Campinas (UNICAMP).
    \item A Liga tem sua sede na Faculdade de Tecnologia da UNICAMP, localizada na Rua Paschoal Marmo, nº 1888, Jardim Nova Itália, Limeira, São Paulo, CEP 13484-332.
\end{enumerate}

\subsection{Art. 2º Do(s) Objetivo(s) e Finalidade(s)}
\begin{enumerate}[label=\S \arabic*.]
    \item A Liga atuará no tripé de ensino, pesquisa e extensão, tendo como finalidade o desenvolvimento, a promoção e a difusão de conhecimentos acerca da área de Cibersegurança, contribuindo para a formação acadêmica e profissional dos alunos a ela vinculados.
    \item São objetivos da Liga:
    \begin{enumerate}[label=\alph*)]
        \item Fomentar a pesquisa, o estudo e a divulgação da Cibersegurança no ambiente acadêmico e na comunidade externa;
        \item Organizar eventos, palestras, workshops, cursos e competições (como CTFs - \textit{Capture The Flag}) para a comunidade acadêmica e demais interessados;
        \item Promover a integração entre estudantes de diferentes cursos, professores e profissionais da área de Cibersegurança;
        \item Desenvolver projetos multidisciplinares que estimulem a aplicação prática dos conhecimentos adquiridos;
        \item Firmar convênios, parcerias e associações com entidades públicas e privadas para atender às suas finalidades e atribuições.
        \item Proporcionar formas progressivas de reconhecimento de aprendizado, como sistema de pontuação.
    \end{enumerate}
\end{enumerate}


\section{Capítulo II – Dos Membros da Liga}


\subsection{Art. 3º Da Classificação dos Membros}
A Liga será composta pelas seguintes categorias de membros:
\begin{enumerate}[label=\alph*)]
    \item \textbf{Membros Fundadores:} Aqueles que participaram da ata de fundação da Liga. Gozam dos mesmos direitos e deveres dos Membros Efetivos.
    \item \textbf{Membros Diretores:} Membros Efetivos eleitos para compor a Diretoria Executiva, com mandato e atribuições definidas neste estatuto.
    \item \textbf{Membros Efetivos:} Discentes da UNICAMP, de graduação ou pós-graduação, que foram aprovados no processo seletivo da Liga e que cumprem com suas obrigações estatutárias.
    \item \textbf{Membros Honorários:} Personalidades, professores ou profissionais que tenham prestado relevantes serviços à Liga ou à área de Cibersegurança. A indicação será feita pela Diretoria Executiva e aprovada em Assembleia Geral. Os Membros Honorários não possuem direito a voto nem obrigação de contribuição financeira.
\end{enumerate}

\subsection{Art. 4º Da Admissão, Direitos e Deveres}
\begin{enumerate}[label=\S \arabic*.]
    \item A admissão de Membros Efetivos ocorrerá por meio de processo seletivo, cujos critérios e periodicidade serão definidos e divulgados pela Diretoria Executiva vigente.
    \item São direitos dos Membros Fundadores e Efetivos:
    \begin{enumerate}[label=\alph*)]
        \item Participar de todas as atividades promovidas pela Liga;
        \item Votar e ser votado para os cargos da Diretoria Executiva, conforme as regras deste estatuto;
        \item Ter acesso às atas de reunião e relatórios financeiros;
        \item Propor projetos e atividades à Diretoria Executiva.
    \end{enumerate}
    \item São deveres de todos os membros:
    \begin{enumerate}[label=\alph*)]
        \item Cumprir e fazer cumprir o presente estatuto;
        \item Zelar pelo bom nome e patrimônio da Liga;
        \item Participar ativamente das atividades para as quais se comprometeu;
        \item Acatar as decisões da Diretoria Executiva e da Assembleia Geral.
    \end{enumerate}
\end{enumerate}

\subsection{Art. 5º Do Desligamento e das Penalidades}
\begin{enumerate}[label=\S \arabic*.]
    \item O desligamento de um membro poderá ocorrer a pedido do próprio, por conclusão de curso na UNICAMP, ou por exclusão.
    \item A exclusão de um membro poderá ser aplicada pela Diretoria Executiva, garantido o direito à ampla defesa, nos seguintes casos:
    \begin{enumerate}[label=\alph*)]
        \item Infringir gravemente as normas deste estatuto;
        \item Praticar atos que desabonem ou prejudiquem a imagem da Liga;
        \item Deixar de cumprir, sem justificativa, com as obrigações assumidas junto à Liga por período superior a 3 (três) meses.
    \end{enumerate}
\end{enumerate}


\section{Capítulo III – Da Estrutura Organizacional}


A Liga é administrada pelos seguintes órgãos:
\begin{enumerate}[label=\alph*)]
    \item Assembleia Geral;
    \item Diretoria Executiva;
    \item Áreas de Atuação;
    \item Comissão Coordenadora (Orientadores);
    \item Conselho Fiscal.
\end{enumerate}

\subsection{Art. 6º Da Diretoria Executiva}
\begin{enumerate}[label=\S \arabic*.]
    \item A Diretoria Executiva é o órgão de administração da Liga, sendo composta por: Presidente(a), Vice-Presidente(a), Secretário(a) e Tesoureiro(a).
    \item O mandato da Diretoria Executiva será de dois (dois) semestres letivos, permitida apenas 1 (uma) reeleição para o mesmo cargo.
    \item Durante os primeiros 4 (quatro) semestres letivos após a fundação da Liga, os Membros Fundadores exercerão os cargos da primeira Diretoria Executiva. Após este período, a composição se dará por eleição direta.
\end{enumerate}

\subsection{Art. 7º Das Áreas de Atuação}
\begin{enumerate}[label=\S \arabic*.]
    \item As atividades da Liga serão organizadas em Áreas de Atuação, cada qual sob a responsabilidade de um ou mais Coordenadores, designados pela Diretoria Executiva.
    \item As áreas são:
    \begin{enumerate}[label=\alph*)]
        \item \textbf{Ensino:} Responsável pela capacitação interna dos membros. Suas atribuições incluem a organização de treinamentos, workshops, grupos de estudo e a criação de materiais didáticos para nivelamento e aprofundamento do conhecimento em Cibersegurança.
        \item \textbf{Externo:} Responsável pela comunicação e relacionamento da Liga com a comunidade externa. Suas atribuições englobam a organização de eventos abertos ao público, a gestão de redes sociais, a prospecção de parcerias com empresas e outras instituições, e a divulgação das atividades da Liga.
        \item \textbf{Projetos:} Responsável pelo desenvolvimento e gerenciamento de projetos práticos em Cibersegurança. Suas atribuições incluem a concepção de desafios (CTFs), o desenvolvimento de ferramentas, a condução de pesquisas aplicadas e o estímulo à participação dos membros em projetos multidisciplinares.
        \item \textbf{Pessoas:} Responsável pela gestão dos membros da Liga. Suas atribuições compreendem a organização dos processos seletivos, a integração de novos membros, a promoção de um ambiente colaborativo e saudável, e o acompanhamento do desenvolvimento e engajamento dos participantes.
    \end{enumerate}
\end{enumerate}

\subsection{Art. 8º Da Comissão Coordenadora}
\begin{enumerate}[label=\S \arabic*.]
    \item A Comissão Coordenadora será composta por, no mínimo, 1 (um) professor da UNICAMP, que atuará como orientador da Liga.
    \item Compete à Comissão Coordenadora orientar academicamente as atividades da Liga, mediar conflitos e ser o elo institucional com a Universidade.
\end{enumerate}

\subsection{Art. 9º Do Conselho Fiscal}
\begin{enumerate}[label=\S \arabic*.]
    \item O Conselho Fiscal será composto por 3 (três) Membros Efetivos não pertencentes à Diretoria Executiva, eleitos juntamente com esta.
    \item Compete ao Conselho Fiscal examinar os balancetes e relatórios financeiros da Liga, emitindo parecer à Assembleia Geral.
\end{enumerate}


\section{Capítulo IV – Das Atribuições da Diretoria}


\subsection{Art. 10º Do Presidente(a)}
Compete ao Presidente(a):
\begin{enumerate}[label=\alph*)]
    \item Representar a Liga judicial e extrajudicialmente;
    \item Convocar e presidir as reuniões da Diretoria Executiva e da Assembleia Geral;
    \item Assinar, em conjunto com o(a) Tesoureiro(a), os documentos financeiros;
    \item Exercer o voto de minerva em caso de empate nas votações da Diretoria;
    \item Coordenar as atividades gerais da Liga e supervisionar as Áreas de Atuação.
\end{enumerate}

\subsection{Art. 11º Do Vice-Presidente(a)}
Compete ao Vice-Presidente(a):
\begin{enumerate}[label=\alph*)]
    \item Substituir o(a) Presidente(a) em suas ausências e impedimentos;
    \item Auxiliar o(a) Presidente(a) em suas atribuições;
    \item Assumir a Presidência em caso de vacância, até o final do mandato.
\end{enumerate}

\subsection{Art. 12º Do(a) Secretário(a)}
Compete ao(à) Secretário(a):
\begin{enumerate}[label=\alph*)]
    \item Redigir e manter em dia as atas das reuniões;
    \item Gerenciar os arquivos, documentos e correspondências da Liga;
    \item Manter atualizada a lista de membros.
\end{enumerate}

\subsection{Art. 13º Do(a) Tesoureiro(a)}
Compete ao(à) Tesoureiro(a):
\begin{enumerate}[label=\alph*)]
    \item Administrar as finanças e o patrimônio da Liga;
    \item Manter em dia a contabilidade e os registros financeiros;
    \item Elaborar relatórios financeiros periódicos e um balanço anual;
    \item Assinar, em conjunto com o(a) Presidente(a), os documentos financeiros.
\end{enumerate}


\section{Capítulo V – Do Patrimônio e das Finanças}


\subsection{Art. 14º Da Receita}
Constituem a receita da Liga:
\begin{enumerate}[label=\alph*)]
    \item Doações, legados e auxílios de pessoas físicas ou jurídicas, públicas ou privadas;
    \item Contribuições de seus membros, se instituídas pela Assembleia Geral;
    \item Rendas provenientes de eventos, cursos ou projetos;
    \item Verbas obtidas através de convênios e parcerias.
\end{enumerate}

\subsection{Art. 15º Da Despesa}
As despesas da Liga deverão se restringir àquelas necessárias para a consecução de seus objetivos, como aquisição de materiais, organização de eventos, e outros custos administrativos, devidamente aprovadas pela Diretoria Executiva.


\section{Capítulo VI – Das Eleições}


\subsection{Art. 16º Do Processo Eleitoral}
\begin{enumerate}[label=\S \arabic*.]
    \item As eleições para a Diretoria Executiva e Conselho Fiscal ocorrerão anualmente, em data a ser definida pela Assembleia Geral.
    \item A convocação para as eleições será feita com antecedência mínima de 30 (trinta) dias.
    \item O processo eleitoral será conduzido por uma comissão eleitoral composta por 3 (três) membros efetivos não candidatos.
    \item A votação será secreta e direta, sendo eleita a chapa que obtiver a maioria simples dos votos dos membros presentes e votantes.
    \item O candidato à eleição para cargo de Diretoria ou Tesouraria deve possuir 2 (dois) ou mais semestres letivos de permanência na Liga, excetuando-se candidatos a cargos de Secretariado.
\end{enumerate}

\subsection{Art. 17º Da Posse}
A posse dos eleitos ocorrerá na primeira reunião ordinária após a proclamação dos resultados da eleição.


\section{Capítulo VII – Das Reuniões}


\subsection{Art. 18º Da Assembleia Geral}
\begin{enumerate}[label=\S \arabic*.]
    \item A Assembleia Geral é o órgão soberano da Liga, composta por todos os membros com direito a voto.
    \item Haverá, no mínimo, uma Assembleia Geral Ordinária por ano para aprovação de contas e eleição da nova diretoria.
    \item Assembleias Gerais Extraordinárias poderão ser convocadas pelo Presidente(a) ou por requerimento de 1/3 (um terço) dos membros efetivos.
\end{enumerate}

\subsection{Art. 19º Das Reuniões da Diretoria}
A Diretoria Executiva se reunirá ordinariamente uma vez por mês, e extraordinariamente sempre que necessário, por convocação do Presidente(a) ou da maioria de seus membros.


\section{Capítulo VIII – Das Disposições Gerais e Finais}


\subsection{Art. 20º Da Reforma do Estatuto}
O presente estatuto só poderá ser reformado em Assembleia Geral Extraordinária, convocada especificamente para este fim, com a aprovação de, no mínimo, 2/3 (dois terços) dos membros presentes.

\subsection{Art. 21º Da Dissolução da Liga}
A dissolução da Liga somente ocorrerá por decisão de uma Assembleia Geral Extraordinária, com aprovação de 2/3 (dois terços) de todos os membros efetivos. Em caso de dissolução, o patrimônio remanescente será destinado a uma entidade congênere ou à Faculdade de Tecnologia da UNICAMP.

\subsection{Art. 22º Dos Casos Omissos}
Os casos omissos neste estatuto serão resolvidos pela Diretoria Executiva, com posterior referendo da Assembleia Geral.

\bigskip
\bigskip
\begin{center}
    Limeira, \today.
\end{center}

\end{document}