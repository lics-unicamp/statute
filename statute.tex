\documentclass[11pt, a4paper]{article}

\usepackage[utf8]{inputenc}
\usepackage[portuguese]{babel}
\usepackage{geometry}
\usepackage{fancyhdr}
\usepackage{hyperref}
\usepackage{enumitem}

\geometry{a4paper,
          top=3cm,
          bottom=2cm,
          left=3cm,
          right=2cm}

\title{Estatuto da Liga de Cibersegurança da UNICAMP}
\author{Diretoria Executiva da Liga de Cibersegurança da UNICAMP}
\date{Versão de \today}

\pagestyle{fancy}
\setlength{\headheight}{14pt}
\fancyhf{}
\fancyhead[L]{Estatuto da Liga de Cibersegurança}
\fancyhead[R]{UNICAMP}
\fancyfoot[C]{\thepage}

\usepackage[scaled]{helvet}
\renewcommand\familydefault{\sfdefault} 
\usepackage[T1]{fontenc}

\begin{document}

\maketitle

\tableofcontents

\newpage

% --- CORPO DO ESTATUTO ---

% ARTIGO I - DA DENOMINAÇÃO, SEDE E OBJETIVOS
\section{Da Denominação, Sede e Objetivos}
\label{sec:denominacao}

\subsection{Das Disposições Preliminares}
\label{sec:denominacao-natureza}
\begin{enumerate}[label=Art. \arabic*.]
    \item A Liga, fundada em \today, caracteriza-se por não ter fins lucrativos, ter duração ilimitada, ser uma sociedade civil, apartidária, não religiosa e vinculada à UNICAMP;
    \item A Liga atuará no estudo, pesquisa e extensão, tendo como finalidade o desenvolvimento, a promoção e a difusão de conhecimentos acerca da área de Cibersegurança, contribuindo para a formação acadêmica e profissional dos alunos a ela vinculados;
    \item Fica a cargo da Liga, através do desenvolvimento de projetos multidisciplinares, promover, facilitar e estimular a integração dos ligantes à área de Cibersegurança;
    \item A Liga poderá firmar convênios e associações com entidades públicas e privadas para atender suas finalidades e atribuições, assim como estabelecer parcerias.
\end{enumerate}


\subsection{Sede}
A sede da Liga se encontra na Faculdade de Tecnologia da UNICAMP, na rua Paschoal Marmo, número 1888, no bairro Jardim Nova Itália, cidade de Limeira, estado de São Paulo.

\subsection{Objetivos}
São objetivos da Liga:
\begin{enumerate}[label=\alph*)]
    \item Fomentar a pesquisa, o estudo e a divulgação da Cibersegurança;
    \item Organizar eventos, palestras e workshops para a comunidade acadêmica;
    \item Promover a integração entre estudantes e profissionais da área.
\end{enumerate}

% ARTIGO II - DOS MEMBROS
\section{Dos Membros}
\label{sec:membros}

\subsection{Categorias de Membros}
A Liga será composta pelas seguintes categorias de membros:
\begin{enumerate}[label=Art. \arabic*.]
    \item Membros Diretores;
    \item Membros Efetivos;
    \item Membros Honorários.
\end{enumerate}

\subsection{Dos Membros Diretores}

\subsection{Dos Membros Efetivos}

\subsection{Dos Membros Honorários}

% ARTIGO III - DA ESTRUTURA ORGANIZACIONAL
\section{Da Estrutura Organizacional}

\subsection{Da Diretoria}
A Liga é administrada por uma Diretoria, eleita a cada \dots.

\subsubsection{Do Presidente}
O Presidente da Diretoria é o responsável por \dots

\subsubsection{Do Vice-Presidente}
O Vice-Presidente será responsável por \dots

\end{document}